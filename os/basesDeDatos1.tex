% Created 2019-03-09 Sat 19:05
\documentclass[11pt]{article}
\usepackage[utf8]{inputenc}
\usepackage[T1]{fontenc}
\usepackage{fixltx2e}
\usepackage{graphicx}
\usepackage{longtable}
\usepackage{float}
\usepackage{wrapfig}
\usepackage{rotating}
\usepackage[normalem]{ulem}
\usepackage{amsmath}
\usepackage{textcomp}
\usepackage{marvosym}
\usepackage{wasysym}
\usepackage{amssymb}
\usepackage{hyperref}
\usepackage[spanish]{babel}   
\usepackage{babel}
\usepackage{apacite}
\usepackage{csquotes}
\tolerance=1000
\author{Jonatan Ahumada Fernández\\
Fundación Universitaria Konrad Lorenz\\
Cod. 506181029}
\date{\today}
\title{Trabajo en bases de datos: \\
\large  Los sistemas POSIX y la memoria NVM/SCM}

\hypersetup{
  pdfkeywords={},
  pdfsubject={},
  pdfcreator={Emacs 25.3.1 (Org mode 8.2.10)}}
\begin{document}
\bibliographystyle{apacite}

\maketitle


La problemática fundamental que el artículo estudia es el desempeño de los sistemas POSIX al usar la memoria NVM/SCM, una tecnología emergente.
Concretamente, se estudia el hecho de que, si un controlador de POSIX intenta gestionar memoria NVM/SCM, le es imposible usar su cualidad más benéfica: que esta es \emph{byte addressable}. 
La motivación de este estudio surge de que, según \cite{article:posix}, el diseño de los primeros sistemas operativos estuvo, en gran medida, constreñido al hardware disponible. En particular, a que no se contaba con
 dispositivos de memoria rápidos, persistentes y \emph{byte addressable}. Esto impulsó unas decisiones de diseño en los sistemas operativos que actualmente impiden que estos se adapten a nuevas tecnologías.
 Así pues, el desempeño subóptimo de la NVM/SCM ejemplifica estas \textquote{profundas} ineficiencias en los sistemas operativos modernos.
La causa fundamental de esto radica en el funcionamiento del \textbf{sistema de archivos} y los \textbf{bloques}  según el estándar POSIX. 
Para comprobar esto, el artículo comparó la ejecución de diferentes aplicaciones en diferentes procesadores utilizando la misma plataforma:
 Ubuntu Linux (3.13.0-24, 3.13.0- 95, 4.2.0-42 kernel versions), utilizando
distintos dispositivos de memoria y distintos sistemas de archivos. Se encontró que un conjunto específico de \textbf{llamadas al sistema} (las relacionadas a la gestión de \emph{metadata})
 ocasionaban el mayor desperdicio de recursos. Por último, se concluyó que el estandar POSIX no puede aprovechar las ventajas que ofrece la memoria NVM y que, por lo tanto, este subconjunto 
de llamadas al sistema se deben optimizar.

El artículo relacionó muchos temas del marco conceptual de la clase y fue posible, a grandes rasgos, comprenderlo gracias a eso. Los temas de intersección con el
curso fueron concretamente: a) dispositivos y gestión de memoria, b) concepto de \emph{kernel}, c) \emph{system calls}, d) gestión de archivos y, por último,
e) historia y evolución de los sistemas operativos. La distinción entre \emph{byte addresable} y \emph{word adressable}, aunque se refieren a temas de gestión de memoria a bajo nivel,
 también fue comprensible, sobre todo si se complementa con los 
capítulos sobre jerarquía de memoria y migración de la información que se encuentran en la bibliografía principal del curso. Personalmente, el artículo capturó mi atención
porque cuestionó, a la luz del surgimiento de nuevas tecnologías, la eficacia del POSIX, un estándar muy reverenciado  entre los programadores "de pura cepa" (por ejemplo, en los movimientos  GNU, Un*x y FreeBSD, todos \emph{POSIX compliant}).

\bibliography{basesDeDatos1}








\end{document}