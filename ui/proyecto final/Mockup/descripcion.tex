% Created 2019-05-13 Mon 17:24
\documentclass[11pt]{article}
\usepackage[utf8]{inputenc}
\usepackage[T1]{fontenc}
\usepackage{fixltx2e}
\usepackage{graphicx}
\usepackage{longtable}
\usepackage{float}
\usepackage{wrapfig}
\usepackage{rotating}
\usepackage[normalem]{ulem}
\usepackage{amsmath}
\usepackage{textcomp}
\usepackage{marvosym}
\usepackage{wasysym}
\usepackage{amssymb}
\usepackage{hyperref}
\tolerance=1000
\author{Jonatan Ahumada Fernández\\
Fundación Universitaria Konrad Lorenz\\
Cód. 506181029}
\date{13 de Mayo de 2019}
\title{Proyecto final:\\
Mockup}
\hypersetup{
  pdfkeywords={},
  pdfsubject={},
  pdfcreator={Emacs 25.3.50.1 (Org mode 8.2.10)}}
 \usepackage{titlesec}
\usepackage{titling}
\usepackage[margin=1in]{geometry}
\titleformat{\section}
{\large \bfseries}
{}
{0em}
{}

\titleformat{\subsection}
{\bfseries}
{}
{0em}
{}[\titlerule]

\titleformat{\subsubsection}
{\bfseries }
{$\bullet$}
{2em}
{}
 
  
\begin{document}

\maketitle




\section{Descripción General}
\label{sec-1}
La aplicación fue diseñada para ser vista en un computador de escritorio. 
Se tuvieron en cuenta los siguientes principios:

\begin{itemize}
\item Cuadrícula de 12 rectángulos ( para que implementar el Bootstrap sea más directo).
\item Cohesión entre las páginas.
\item Simplicidad en las acciones y en el despliegue de la información.
\item Cada funcionalidad --consulta, registro, eliminación-- fue separada.
\end{itemize}

\section{Nota sobre la implementación}
\label{sec-2}
Como no tengo conocimiento sobre Bases de Datos, implementaré todo como páginas estáticas. Los valores estarán 
guardados en el Javascript. 


\section{Resumen de clases}
\label{sec-3}
\begin{itemize}
\item divHeader
\item divFooter
\item h1
\item h2
\item formButton
\item BacktoMenuButton
\item confirmationParagraph
\item errorParagraph
\item normalParagrpah
\end{itemize}
% Emacs 25.3.50.1 (Org mode 8.2.10)
\end{document}