% Created 2019-02-23 Sat 19:12
\documentclass[11pt]{article}
\usepackage[utf8]{inputenc}
\usepackage[T1]{fontenc}
\usepackage{fixltx2e}
\usepackage{graphicx}
\usepackage{longtable}
\usepackage{float}
\usepackage{wrapfig}
\usepackage{rotating}
\usepackage[normalem]{ulem}
\usepackage{amsmath}
\usepackage{textcomp}
\usepackage{marvosym}
\usepackage{wasysym}
\usepackage{amssymb}
\usepackage{hyperref}
\tolerance=1000
\author{Jonatan Ahumada Fernández}
\date{\textit{<2019-02-23 Sat>}}
\title{Cálculo Integral: Repaso}
\hypersetup{
  pdfkeywords={},
  pdfsubject={},
  pdfcreator={Emacs 25.3.1 (Org mode 8.2.10)}}
\begin{document}

\maketitle
\tableofcontents


\section{Álgebra}
\label{sec-1}

\subsection{Teorema del binomio}
\label{sec-1-1}
Para el binomio \( (a + b)^n\), su expansión tiene las sigientes características
\begin{enumerate}
\item En total habrá n+1 términos
\item Por cada término, el poder de a decremente y el de a incrementa.
\item Cada término tiene la forma \((c)a^{n-k}{b^k}\), donde c es un entero y \(k = 0,1,2, \dots n\)
\item La siguiente fórmula se cumple para cada uno de los n términos de la expansión se cumple que: \(\frac{coeficiente\ de\ término\ x\ potencia\ de\ término}{número\ término} = coeficiente\ del\ siguiente\ término\)
\end{enumerate}

\subsection{Resta al cuadrado}
\label{sec-1-2}
El último término siempre da positivo y el del medio negativo:

\[(a-b)^2 = a^2 -2ab + b^2\]   


\section{Sustitución}
\label{sec-2}
\subsection{Guías para sustitución simple}
\label{sec-2-1}
\begin{enumerate}
\item Identifica si \(u\) y \(du\) aparecen en la misma expresión (salvo una diferencia de constantes).
\item Sustituye lo más complejo. Después integrar y derivar sus exponentes será más fácil.
\item Identifica qué identidad trigonométrica usarás.
\end{enumerate}

\subsection{Guías para sustitución por partes}
\label{sec-2-2}
\begin{enumerate}
\item Se usa cuando las funciones implicadas no tienen relación en términios de sus derivadas (no hay \(u\) y \(du\))
\item Ten claro el acrónimo ALPES antes de seleccionar \(u\) y \(dv\).
\item Como aquí toca derivar, no olvides regla de la cadena.
\item Aquí se tienen en cuenta las potencias al reemplazar.
\end{enumerate}


\begin{verbatim}
+--------------------------+
|                          |
|         derivar          |
|    u  --------->    du   |
|        integrar          |
|    dv -------->     v    |
|                          |
|\int[funcion]dx=u-\intdu*v|   
+--------------------------+
\end{verbatim}
\section{Trigonometría}
\label{sec-3}
\subsection{Integrales fundamentales}
\label{sec-3-1}
\[\int \cos(mx)dx =  \frac{1}{m} \sin(mx)\]
\[\int \sin(mx)dx = - \frac{1}{m} \cos(mx)\]
\subsection{Identidad fundamental (potencias impares)}
\label{sec-3-2}
\[\sin^2(x) + \cos^2(x) = 1\]

Consistirá en expresar una expresión trigonometrica impar en términos de una par.
Luego, se reemplazará una función trigonometrica al cuadrado por su identidad.

Luego, de sustituir, \(u\) y \(du\), se resolverá un binomio cuadrado.







\subsection{Ángulo Medio (potencias pares)}
\label{sec-3-3}

Tener cuidado con los signos.

\[\sin^2(x) = \frac{1 - \cos(2x)}{2}  \]
El coeficiente de x se duplica.

\[\cos^2(x) = \frac{1 + \cos(2x)}{2}  \]

\subsubsection{Variación}
\label{sec-3-3-1}

\[2\cos^2(x) = 1 + \cos(2x) \]

Es lo mismo, solamente pasa el dos al otro lado.
Como es el proceso inverso, el coeficiente de x se divide 
y crece la potencia. 

\subsection{Eliminación de raíces}
\label{sec-3-4}
\subsubsection{Con ángulo medio}
\label{sec-3-4-1}

\[\int \sqrt{\frac{1 - \cos(2x)}{2}} dx = \int \sqrt{\sin^2(x)}dx \]

\subsubsection{Con variación}
\label{sec-3-4-2}
\[\int \sqrt{1 - \cos(2x)} dx = \int \sqrt{2\cos^2(x)}dx \]
\subsection{Integrales capciosas}
\label{sec-3-5}


\begin{center}
\begin{tabular}{ll}
Integral & Expresión\\
\hline
$\int$ $\ln$(x) dx & $\frac{1}{x}$\\
$\int$ e$^{\text{cx}}$ dx & $\frac{e^{cx}}{c}$\\
$\int$ tan(x) dx & $\sec^2(x)$\\
$\int$ sec(x) dx & $sec(x)tan(x)$\\
\end{tabular}
\end{center}
% Emacs 25.3.1 (Org mode 8.2.10)
\end{document}