El artículo seleccionado en esta ocasión es \emph{Center of Mass of a
Soft Spring}, el cual se comparará con \emph{Cutting Against the Grain:
Volumes of Solids of Revolution via Cross Sections Parallel to the
Rotation Axis}. Ambos artículos tienen por objetivo mostrar el valor
pedagógico, a nivel de pregrado, de una solución poco común a un
problema canónico. Para esto, ambos artículos utilizan las herramientas
del cálculo. En el artículo de Benedetto, se halla el volúmen de un
sólido de revolución por secciones paralelas; por su parte, (D. Serna \&
Amhitab) halla el centro de masa de un resorte "suave" extendido
verticalmente de forma no uniforme, por medio de conceptos bastante
elementales: sumas infinitesimales y longitud de arco.

Si bien ambos artículos buscan mostrar al estudiante las aplicaciones
del cálculo de forma no canónica, el artículo de Benedetto tuvo el mayor
impacto en mí. En primer lugar, D. Serna \& Amhitab delimitan su
problema a un caso bastante particular, a saber, el centro de mas de un
resorte supendido verticalmente que se elonga por acción de su propio
peso. Además, su solución tiene ciertas discrepancias con resultados
obtenidos con otros enfoques. Esto contrasta con el problema del Cuerno
de Gabriel, que consta de mucha mayor elegancia, puesto que
aparentemente entraña una antinomia: una figura con una superficie de
área infinita, pero con volumen finito. En segundo lugar, el Cuerno de
Gabriel parte de una función sencilla \[\frac{1}{x}\]. En este sentido,
la solución de Benedetto cubre el caso más general de todas las
funciones asintóticas a las cuales se les quiera calcular el volúmen
correspondiente a su sólido por secciones transversales. Esta
simplicidad me parece más didáctica, puesto que permite centrarse en la
matemática.

Por último, Benedetto estructuró mejor su artículo. Resultó gratificante
ver cómo, lo que en principio no era más que una curva en el plano,
gradualmente crecía en dificultad al intentar rotarla sobre el eje x.
Esto requirió retomar temas vistos en cursos anteriores (volumen de un
cono, hipérbola, parametrización) de forma aislada y ahora utilizarlos
de una forma intuitiva. En síntesis, el artículo de Benedetto unió lo
visto hasta el momento y, además, dejó promesas aún más grandes. Frente
a esto, el artículo de D. Serna \& Amhitab parece meramente anecdótico.
