% Created 2019-03-17 Sun 19:49
\documentclass[11pt]{article}
\usepackage[utf8]{inputenc}
\usepackage[T1]{fontenc}
\usepackage{fixltx2e}
\usepackage{graphicx}
\usepackage{longtable}
\usepackage{float}
\usepackage{wrapfig}
\usepackage{rotating}
\usepackage[normalem]{ulem}
\usepackage{amsmath}
\usepackage{textcomp}
\usepackage{marvosym}
\usepackage{wasysym}
\usepackage{amssymb}
\usepackage{hyperref}
\usepackage{apa}
\tolerance=1000
\author{Jonatan Ahumada Fernández\\
Fundación Universitaria Konrad Lorenz\\
Cód. 506181029}
\date{\today}
\title{Trabajo en bases de datos I:\\
Un caso especial de sólidos de revolución}
\hypersetup{
  pdfkeywords={},
  pdfsubject={},
  pdfcreator={Emacs 25.3.1 (Org mode 8.2.10)}}
\begin{document}

\maketitle



\section{Código asignado en el MSC }
\label{sec-1}
El tema “integral calculus” tiene el código 97I50. 


\section{Artículos preseleccionados}
\label{sec-2}
Los artículos seleccionados fueron \emph{Cutting Against the Grain: Volumes of Solids of Revolution via Cross-Sections Parallel to the Rotation Axis} \cite{knudson},
\emph{Using origami boxes to explore concepts of geometry and calculus} \cite{wares} y \emph{Some remarks about flux time derivative} \cite{benedetto}. 

\section{Resumen del artículo}
\label{sec-3}

El artículo explica un caso especial para calcular el volumen de un sólido de revolución.
Normalmente, este se calcula sumando el área de las secciones perpendiculares al 
eje de rotación. No obstante,  en el caso estudiado, se suman las secciones \textbf{paralelas} al eje de rotación, en vez de las secciones 
perpendiculares a él. El ejemplo escogido para explicar este caso es el "Cuerno de Gabriel" (la gráfica 
de la función \(y = 1/x\)).
\section{Secciones del artículo}
\label{sec-4}
\subsection{Sólidos de revolución}
\label{sec-4-1}
En esta sección se explica cómo parametrizar la superficie del sólido suponiendo que 
lo seccionamos en el plano \(z=c\). Desde esta perspectiva, cada sección del Cuerno se ve 
como una área delimitada por la curva de la función. Así que, conociendo la ecuación de la
curva, podemos calcular el área de la sección como una integral. 

\subsection{El volumen de un cono}
\label{sec-4-2}
En esta sección se explica cómo utilizar el resultado de la sección anterior para determinar el volumen
de un sólido básico: el cono. Los resultados anteriores lograban expresar el área de una sección 
en términos de una integral, pero los límites (\emph{bounds}) de esta integral eran indefinidos. Para hallar 
exactamente los límites de esta integral, se estudia el comportamiento de los límites de una curva muy
familiar: la hiperbola. 

\subsection{Una integral impropia}
\label{sec-4-3}
Antes de aplicar finalmente los resultados al caso en cuestión, se hace un último paréntesis
y se considera el cálculo del volumen de la gráfica de la función \(y= e^{-x}\). El problema estudiado aquí, 
en términos muy llanos, es cómo lograr integrar la rotación de las áreas de las hipérbolas obtenidas previamente. 

\subsection{El Cuerno de Gabriel}
\label{sec-4-4}
En esta parte se juntan todos las conclusiones previas para efectivamente calcular 
el volumen del Cuerno de Gabriel. En esta sección simplemente se aplican las formulas deducidas
a la función  \(y= 1/x\). Un breve resumen podría ser: 1) parametrización, 2)  hallar 
los límites de la curva, 3) hacer las sustituciones necesarias para encontrar la forma de 
rotar la integral. 

El autor concluye que estudiar este problema de esta forma "no convencional" es muy útil
para comprender una "paradoja" conceptual. Según el autor, si se secciona el Cuerno de esta forma
vemos que su área superficial es infinita (porque estamos viendo el cono como muchas hiperbolas, en vez 
de verlo como si fueran muchos cilindros), y, a pesar de eso, logramos obtener un volumen finito. 
Así, el autor le atribuye a este hecho gran valor didáctico. 

\subsection{Sugerencias para estudios posteriores}
\label{sec-4-5}
En esta sección el autor menciona otros posibles ejemplos donde aplicar 
esta técnica. Entre ellos, menciona las funciones \(y= \sqrt{x}\)  \(y= \sqrt[4]{x}\)  \(y = \sqrt{ln x}\).

\bibliography{bibliografía}
% Emacs 25.3.1 (Org mode 8.2.10)

\end{document}