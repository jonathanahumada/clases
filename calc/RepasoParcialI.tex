% Created 2019-04-21 Sun 17:24
\documentclass[11pt]{article}
\usepackage[utf8]{inputenc}
\usepackage[T1]{fontenc}
\usepackage{fixltx2e}
\usepackage{graphicx}
\usepackage{longtable}
\usepackage{float}
\usepackage{wrapfig}
\usepackage{rotating}
\usepackage[normalem]{ulem}
\usepackage{amsmath}
\usepackage{textcomp}
\usepackage{marvosym}
\usepackage{wasysym}
\usepackage{amssymb}
\usepackage{hyperref}
\usepackage{multicol}
\usepackage{titlesec}
\usepackage[margin=0.5in]{geometry}


\titleformat{\section}
{\huge \bfseries}
{}
{0em}
{}

\titleformat{\subsection}
{\bfseries}
{}
{0em}
{}[\titlerule]

\titleformat{\subsubsection}
{\bfseries }
{$\bullet$}
{2em}
{}
\tolerance=1000
\author{Jonatan Ahumada Fernández}
\date{\textit{<2019-02-23 Sat>}}
\title{Cálculo Integral: RepasoII}
\hypersetup{
  pdfkeywords={},
  pdfsubject={},
  pdfcreator={Emacs 25.3.1 (Org mode 8.2.10)}}
\begin{document}
\begin{multicols}{3}


\section{Sustitución}
\label{sec-1}
\subsection{Sustitución simple}
\label{sec-1-1}
\begin{enumerate}
\item Identifica si \(u\) y \(du\) aparecen en la misma expresión (salvo una diferencia de constantes).
\item Sustituye lo más complejo. Después integrar y derivar sus exponentes será más fácil.
\item Identifica qué identidad trigonométrica usarás.
\end{enumerate}

\subsection{Sustitución por partes}
\label{sec-1-2}
\begin{enumerate}
\item Se usa cuando las funciones implicadas no tienen relación en términios de sus derivadas (no hay \(u\) y \(du\))
\item Ten claro el acrónimo ALPES antes de seleccionar \(u\) y \(dv\).
\item Como aquí toca derivar, no olvides regla de la cadena.
\item Aquí se tienen en cuenta las potencias al reemplazar.
\end{enumerate}


\begin{verbatim}
+--------------------------+
|                          |
|         derivar          |
|   	u  --------->   du    |
|        integrar          |
|    dv -------->     v    |
|                          |
|\int[u*dv]dx=u*v-\intdu*v |   
+--------------------------+
\end{verbatim}
\section{Trigonometría}
\label{sec-2}
\subsection{Integrales fundamentales}
\label{sec-2-1}
\[\int \cos(mx)dx =  \frac{1}{m} \sin(mx)\]
\[\int \sin(mx)dx = - \frac{1}{m} \cos(mx)\]
\subsection{Identidad fundamental (potencias impares)}
\label{sec-2-2}
\[\sin^2(x) + \cos^2(x) = 1\]

Consistirá en expresar una expresión trigonometrica impar en términos de una par.
Luego, se reemplazará una función trigonometrica al cuadrado por su identidad.

Luego, de sustituir, \(u\) y \(du\), se resolverá un binomio cuadrado.







\subsection{Ángulo Medio (potencias pares)}
\label{sec-2-3}

Tener cuidado con los signos.

\[\sin^2(x) = \frac{1 - \cos(2x)}{2}  \]
El coeficiente de x se duplica.

\[\cos^2(x) = \frac{1 + \cos(2x)}{2}  \]

\subsubsection{Variación}
\label{sec-2-3-1}

\[2\cos^2(x) = 1 + \cos(2x) \]

Es lo mismo, solamente pasa el dos al otro lado.
Como es el proceso inverso, el coeficiente de x se divide 
y crece la potencia. 

\subsection{Eliminación de raíces}
\label{sec-2-4}
\subsubsection{Con ángulo medio}
\label{sec-2-4-1}

\[\int \sqrt{\frac{1 - \cos(2x)}{2}} dx = \int \sqrt{\sin^2(x)}dx \]

\subsubsection{Con variación}
\label{sec-2-4-2}
\[\int \sqrt{1 - \cos(2x)} dx = \int \sqrt{2\cos^2(x)}dx \]
\subsection{Integrales capciosas}
\label{sec-2-5}


\begin{center}
\begin{tabular}{ll}
Integral & Expresión\\
\hline
$\int$ $\ln$(x) dx & $\frac{1}{x}$\\
$\int$ e$^{\text{cx}}$ dx & $\frac{e^{cx}}{c}$\\
$\int$ tan(x) dx & $\sec^2(x)$\\
$\int$ sec(x) dx & $\sec(x)tan(x)$\\
\end{tabular}
\end{center}


\subsection{Sustitución trigonométrica}
\label{sec-2-6}


\begin{center}
\begin{tabular}{ll}
Caso & Expresión\\
\hline
$\int \sqrt{a^2 -x^2} dx$ & $x = a \sin \theta$\\
$\int \sqrt{a^2 + x^2} dx$ & $x = a \tan \theta$\\
$\int \sqrt{x^2 -a^2} dx$ & $x = a \sec \theta$\\
\end{tabular}
\end{center}

\begin{center}
\begin{tabular}{ll}
Identidades específicas & \\
\hline
$\sin(2\theta$) & $2 \sin\theta \cos\theta$\\
$\cos(2\theta)$ & $\cos^2\theta - \sin^2\theta$\\
\end{tabular}
\end{center}

\subsection{Tablas trignométricas}
\label{sec-2-7}
\begin{center}
\begin{tabular}{rllll}
$\theta$ & radianes & $\sin \theta$ & $\cos \theta$ & $\tan \theta$\\
\hline
0 & 0 & 0 & 1 & 0\\
30 & $\frac{\pi}{6}$ & $\frac{1}{2}$ & $\frac{\sqrt{3}}{2}$ & $\frac{\sqrt{3}}{3}$\\
45 & $\frac{\pi}{4}$ & $\frac{\sqrt{2}}{2}$ & $\frac{\sqrt{2}}{2}$ & 1\\
60 & $\frac{\pi}{3}$ & $\frac{\sqrt{3}}{2}$ & $\frac{1}{2}$ & $\sqrt{3}$\\
90 & $\frac{\pi}{2}$ & 1 & 0 & --\\
\end{tabular}
\end{center}

\subsection{Fracciones Parciales}
\label{sec-2-8}

\begin{center}
\begin{tabular}{ll}
Caso & Fórmula\\
\hline
lineal irrepetible & $\frac{A}{(ax+b)}$\\
lineal repetible & $\frac{A_{n}}{(ax+b)^n}$\\
cuadrática irrepetible & $\frac{Ax + B}{(ax+b)}$\\
cuadrática repetible & $\frac{Ax + B}{(ax+b)^n}$\\
\end{tabular}
\end{center}

\subsubsection{Siempre simplifica el denominador}
\label{sec-2-8-1}

\subsection{Teorema Fundamental}
\label{sec-2-9}
\[ \frac{dF}{dx} = \frac{d}{dx}\int_{a}^{x}f(t)dt=f(x)\]

\subsection{Promedio}
\label{sec-2-10}
\[\frac{1}{b-a} \int_{a}{b}f(x)dx\]
\section{Volúmenes}
\label{sec-2-11}
\begin{center}
\begin{tabular}{ll}
Caso & Fórmula\\
\hline
S. trans. & $\int_{a}^{b} Base \times Altura dx$\\
Cilindros & $\int_{a}^{b}\pi R^2 dx$\\
Arandelas & $\int_{a}^{b}\pi(R^2 - r^2) dx$\\
C. cilíndr. & $\int_{a}^{b}2\pi x f(x)  dx$\\
\end{tabular}
\end{center}

\subsection{Aplicaciones}
\label{sec-2-12}

\begin{center}
\begin{tabular}{ll}
L. arco & $\int_{a}^{b} \sqrt{1 + \frac{dy}{dx}^2}$ dx\\
A. s. rev. & $\int_{a}^{b} 2\pi f(x) \sqrt{1 + \frac{dy}{dx}^2}$ dx\\
Hooke & $W=\int_{a}^{b} F(x)dx$, $F(x)=kx$\\
Momento($M_{0}$) & $\int_{a}^{b}x\delta(x)dx$\\
Masa ($M$) & $\int_{a}^{b}\delta(x)dx$\\
C. masa & $\frac{M_{0}}{M}$\\
\end{tabular}
\end{center}



\section{Tablas}
\label{sec-2-13}

\begin{center}
\begin{tabular}{ll}
Derivadas & \\
\hline
$\sin x$ & $\cos x$\\
$\cos x$ & $-\sin x$\\
$\tan x$ & $\sec^2 x$\\
$\cot x$ & $-\csc^2 x$\\
$\sec x$ & $\sex x \tan x$\\
$\csc x$ & $-\csc x \cot x$\\
\end{tabular}
\end{center}

\begin{center}
\begin{tabular}{ll}
Integrales & \\
\hline
$\sin x$ & $- \cos x + x$\\
$\cos x$ & $\sin x + c$\\
$\tan x$ & $-\ln(cos x) + c$\\
$\cot x$ & $\ln(\sin x) + c$\\
$\sec x$ & $\ln(\sec x + \tan x) + c$\\
$\csc x$ & $\ln(\csc x - \cot x) + c$\\
\end{tabular}
\end{center}


\begin{center}
\begin{tabular}{ll}
Casos notables & \\
\hline
$\int \tan^2 \theta d\theta$ & $\int \sec^2 \theta - 1 d\theta$\\
\end{tabular}
\end{center}
\end{multicols}
\end{document}