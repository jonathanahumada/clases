% Created 2019-05-12 Sun 15:50
\documentclass[11pt]{article}
\usepackage[utf8]{inputenc}
\usepackage[T1]{fontenc}
\usepackage{fixltx2e}
\usepackage{graphicx}
\usepackage{longtable}
\usepackage{float}
\usepackage{wrapfig}
\usepackage{rotating}
\usepackage[normalem]{ulem}
\usepackage{amsmath}
\usepackage{textcomp}
\usepackage{marvosym}
\usepackage{wasysym}
\usepackage{amssymb}
\usepackage{hyperref}
\usepackage{apacite}
\bibliographystyle{apacite}
\usepackage[utf8]{inputenc}
\usepackage[spanish]{babel}

\tolerance=1000
\author{Jonatan Ahumada\\
Fundación Universitaria Konrad Lorenz\\
Cód. 506181029}
\date{14 de Mayo de 2019}
\title{Trabajo en base de datos II \\
}
\hypersetup{
  pdfkeywords={},
  pdfsubject={},
  pdfcreator={Emacs 25.3.50.1 (Org mode 8.2.10)}}
  \usepackage{titlesec}
\usepackage{titling}
\usepackage[margin=0.7in]{geometry}
\titleformat{\section}
{\large \bfseries}
{}
{0em}
{}

\titleformat{\subsection}
{\bfseries}
{}
{0em}
{}[\titlerule]

\titleformat{\subsubsection}
{\bfseries }
{$\bullet$}
{2em}
{}

\begin{document}



\maketitle
\section{}
El artículo seleccionado en esta ocasión es \emph{The Center of Mass of a Soft Spring} \cite{serna}, el cual 
se comparará con \emph{Cutting Against the Grain: Volumes of Solids of Revolution via Cross Sections Parallel to the Rotation Axis} \cite{knudson}.
Ambos artículos son muy similares en tanto que muestran el valor pedagógico de hallar una solución distinta a la canónica para un problema dado.
 Para esto, ambos artículos utilizan las herramientas del cálculo. En el artículo de Knudson, se halla el volumen de un sólido
de revolución por secciones paralelas; por su parte, Serna \& Joshi hallan el centro de masa de un resorte "suave" extendido
verticalmente de forma no uniforme, por medio de conceptos bastante elementales: sumas infinitesimales y longitud de arco. 

Si bien ambos artículos buscan mostrar al estudiante las aplicaciones del cálculo de forma no canónica, el artículo de Knudson
tuvo el mayor impacto en mí. En primer lugar, Serna \& Joshi delimitan su problema a un caso bastante particular (solo aplica a un tipo de resorte y sólo en caída libre). Además, su solución tiene ciertas
discrepancias con resultados obtenidos con otros enfoques. Esto contrasta con el problema del Cuerno de Gabriel,
que consta de mucha mayor elegancia, puesto que aparentemente entraña una antinomia: una figura con una superficie de área infinita, pero con
volumen finito. En segundo lugar, el Cuerno de Gabriel es una función sencilla ($\frac{1}{x}$). En este sentido, la solución de Knudson
cubre el caso más general de todas  las funciones asintóticas (o muchas) a las cuales se les quiera calcular el volumen correspondiente a su sólido
por secciones transversales. Esta simplicidad me parece más didáctica, puesto que permite centrarse en la matemática. 

Por último, Knudson estructuró mejor su artículo. Resultó gratificante ver cómo
 lo que, en principio, no era más que una curva en el plano gradualmente crecía en dificultad al intentar rotarla sobre el eje x.
Esto requirió retomar temas vistos en cursos anteriores (volumen de un cono, hipérbola, parametrización) de forma aislada y
ahora utilizarlos de una forma intuitiva. En síntesis, el artículo de Knudson unió lo visto hasta el momento y, además, dejó 
promesas aún más grandes. Frente a esto, el artículo de Serna \& Joshi parece meramente anecdótico.  

\bibliography{referenciasII}
% Emacs 25.3.50.1 (Org mode 8.2.10)
\end{document}