% Created 2019-05-20 Mon 13:15
\documentclass[11pt]{article}
\usepackage[utf8]{inputenc}
\usepackage[T1]{fontenc}
\usepackage{fixltx2e}
\usepackage{graphicx}
\usepackage{longtable}
\usepackage{float}
\usepackage{wrapfig}
\usepackage{rotating}
\usepackage[normalem]{ulem}
\usepackage{amsmath}
\usepackage{textcomp}
\usepackage{marvosym}
\usepackage{wasysym}
\usepackage{amssymb}
\usepackage{hyperref}
\usepackage[spanish]{babel}
\usepackage{titlesec}
\usepackage{titling}
\usepackage[margin=1.5in]{geometry}
\titleformat{\section}
{\large \bfseries}
{}
{0em}
{}

\titleformat{\subsection}
{\bfseries}
{}
{0em}
{}[\titlerule]

\titleformat{\subsubsection}
{\bfseries }
{$\bullet$}
{2em}
{}

\tolerance=1000
\author{Jonatan Ahumada Fernández\\
Fundación Universitaria Konrad Lorenz\\
Cód. 506181029}
\date{\today}
\title{Trabajo en Bases de Datos II}
\hypersetup{
  pdfkeywords={},
  pdfsubject={},
  pdfcreator={Emacs 25.3.50.1 (Org mode 8.2.10)}}
\begin{document}

\maketitle
\noindent
El objetivo del artículo es proponer un nuevo método para simular un sistema distribuido. La particularidad
del enfoque de este artículo es que propone hacerlo mediante un concepto denominado "sistemas multi-agente".
La diferencia de este método, con respecto a otras formas de abordar los sistemas distribuidos, es que en él se implementa
el concepto de \emph{autoorganización}, el cual se inspira en el comportamiento de algunos insectos sociales. Según los autores, 
en un sistema multi-agente, cada componente tiene la capacidad de actuar de forma autónoma e independiente, pero de tal manera
que al interactuar con los otros componentes se cumpla un objetivo global. Esta propiedad es de especial importancia para los autores, 
puesto que señalan que un posible uso de esta propiedad es generar robustez y auto-recuperación de fallas en un sistema distribuido, 
en virtud de que cada componente es independiente de los demás. El problema en especifico que se utiliza para demostrar esta simulación es el de explorar un espacio de forma eficiente.
Más concretamente, un plano bidimensional.

La relación de este artículo con estructuras de datos se evidencia a lo largo de toda la modelación de la simulación. De forma 
más general, las dos objetos principales dentro de la simulación son los \emph{agentes móviles} y los \emph{agentes estáticos}. Estos objetos
 se pueden considerar estructuras de datos, puesto que las estructuras de datos que hemos estudiado en el curso 
se implementan como objetos en Java. No obstante, hay otros momentos cruciales donde se evidencia el uso de estructuras de datos. Por ejemplo, 1)
en la definición del comportamiento base de los agentes y 2) en la definición de los procesos para que el sistema se repare a sí mismo  en caso de falla.
En los casos recién mencionados, se brinda una definición en pseudocódigo de los procedimientos, muy similar a los ADT
estudiados en el curso. 

Por último, la contribución del uso de temas de estructuras de datos en el artículo ayudó a comprender los conceptos de \emph{sistemas distribuidos}, \emph{sistemas multi-agente}  y redes, puesto que estas se definieron en el artículo como una colección de nodos y agentes, una definición formal 
similar, por ejemplo, a una lista enlazada (Chain, ChainNode). Esto ayudó a formar una comprensión suficiente de lo discutido en el artículo, sin tener que entrar en detalles sobre qué es sistema distribuido. Así mismo, 
el comportamiento de los agentes con respecto al entorno y sus vecinos fue muy similar al concepto de vecindad utilizado en el ejemplo de búsquedas en profundidad visto en clase. 
En síntesis, el artículo permitió relacionar lo visto en clase con un posible uso de modelado de un problema real. 


\section{Referencias}
\label{sec-1}
\ \ Portela, A. R \& Gómez, J. (2019). "Simulación de comportamientos de sistemas distribuidos para obtener robustez y auto-recuperación de fallas": Diálogos sobre investigación: avances científicos Konrad Lorenz. Gustavo Andrés Campos Avendaño, María  Alejandra Castaño Hernández, Mercedes Gaitán Ángulo y Vanessa Sánchez Mendoza (Compiladores); Laura Acelas Russi\ldots{} [et al.]. Bogotá: Fundación Universitaria Konrad Lorenz, 2019, 217-243. 
% Emacs 25.3.50.1 (Org mode 8.2.10)
\end{document}