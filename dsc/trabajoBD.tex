% Created 2019-03-16 Sat 14:36
\documentclass[11pt]{article}
\usepackage[utf8]{inputenc}
\usepackage[T1]{fontenc}
\usepackage{fixltx2e}
\usepackage{graphicx}
\usepackage{longtable}
\usepackage{float}
\usepackage{wrapfig}
\usepackage{rotating}
\usepackage[normalem]{ulem}
\usepackage{amsmath}
\usepackage{textcomp}
\usepackage{marvosym}
\usepackage{wasysym}
\usepackage{amssymb}
\usepackage{hyperref}
\usepackage{cite}

\tolerance=1000
\author{Jonatan Ahumada}
\date{\today}
\title{Trabajo en bases de datos:\\
Un modelo discreto para modelar la compra compulsiva}
\hypersetup{
  pdfkeywords={},
  pdfsubject={},
  pdfcreator={Emacs 25.3.1 (Org mode 8.2.10)}}
\begin{document}

\maketitle


\bibliographystyle{apacite}
\section{Detalles de la publicación}
\label{sec-1}

El artículo "A discrete mathematical model for addictive buying: Predicting
the affected population evolution" fue publicado en la revista \emph{Mathematical and
Computing Modelling}, en los números 7 y 8 de su volúmen 54 (publicado en el 2011).
Sus autores son I. García , L. Jódar, P. Merello, F. -J. Santonja, todos provenientes
de distintas universidades de España. El artículo ha sido citado en otros 10 trabajos según
el índice Scopus, aunque solo se han registrado 9 lecturas según PlumX Metrics.

\section{Resumen del artículo}
\label{sec-2}

El artículo se divide en dos partes. En la primera, se busca determinar la "prevalencia 
de los compradores compulsivos" en España para el año 2010. Para poder determinar esto,
 se realizaron dos encuestas distintas (una en el 2001 y otra en el 2010) en Vizcaya (España) sobre los hábitos de compra de los encuestados. En esta primera parte, fue crítico 
extraer una muestra representativa de estas dos encuestas utilizando un \textbf{muestreo stratificado}, para 
asegurar la representación de diversos grupos de edades y según el sexo. Una vez determinada la muestra, 
se clasificó a los individuos utilizando una "escala de compra compulsiva" que categoriza a los individuos
en 1) compradores racionales, 2) compradores excesivos y 3) compradores adictos. 

En la segunda parte, luego de encontrar los valores para estas 3 variables, el artículo propuso un modelo matemático
que las relacionaba en función de la población total. Se señala que el modelo propuesto
es discreto porque los autores consideran que el comportamiento del consumidor puede ser considerado como una sucesión
de \emph{steady states} que ocasionalmente presentan saltos de un estado a otro. Gracias a este modelo matemático, se hace una predicción
para los tres años siguientes. Además, el modelo permite hacer predicciones según los tipos de escenarios planteados, ya que solo 
se deben cambiar los valores de ciertas variables en la "dinámica" del modelo (por ejemplo, el porcentaje de la población que tiene una opinión
optimista sobre la situación). El estudio consideró 
4 escenarios distintos de proyección, en todos ellos la prevalencia de compradores compulsivos aumenta.


\section{Discusión}
\label{sec-3}
El artículo propone una herramienta matemática para analizar el aumento en compradores compulsivos en España, la cual es 
considerada una "epidemia". El artículo menciona que su motivacion ha sido la dificultad de predecir el 
comportamiento de los compradores, un fenómeno que involucra muchas variables psicológicas y sociales. Según los autores, nunca se 
ha hecho modelo que permita establecer concretamente la "prevalencia" de ese comportamiento para años venideros, a pesar de que se han hecho trabajos 
desde otros enfoques. A mi entender, el hecho de que el problema se plantea de manera discreta, es decir, bajo el supuesto de que el comportamiento 
de un comprador súbitamente cambia de un estado a otro, ayudó mucho a plantear y desarrollar este nuevo enfoque.

La conclusión del artículo puede parecer un poco obvia: que la prevalencia de compradores compulsivos en España, en los 3 años posteriores a la 
muestra (hasta el 2013), crecerá sin importar los 4 escenarios donde se proyecte. No obstante, más allá de eso, los autores recalcan que es un logro 
poder aplicar un modelo matemático para estudiar las epidemias sociales, pues permiten conocer sus dinámicas y simular distintos resultados según 
los escenarios que se quieran considerar. 

El punto de encuentro con el curso se dió en el tema de sucesiones. El modelo matemático propuesto fue expresado como una
sucesión muy elemental \(P_{n}= N_{n} + S_{n} + A_{n}\), donde n es el tiempo en meses, N es el número de compradores racionales, S, el de exesivos y 
A, el de compulsivos.  Luego, la \textbf{dinámica} de la población fue descrita en términos de un sistema de ecuaciónes, que me pareció muy 
familiar el método para resolver relaciones de recurrencia lineales homogéneas (ambos utilizan sistemas de ecuaciones). No obstante, el concepto de \textbf{dinámica}
no me fue muy claro, a pesar de que noto esta similaridad. 

Mi opinión sobre el artículo está mezclada. Por un lado, fue muy ilustrativo ver el proceso de cómo formular un modelo matemático,
 utilizando las herramientas vistas en clase para analizar una muestra
estadística. Esa fue mi mayor motivación al seleccionarlo. 
Desafortunadamente, supone que el lector
 ya domina los conceptos empleados (por ejemplo, 
muestro estratificado, \emph{cluster analysis}, \emph{dynamic}). Esto da la impresión de que el artículo no elabora muy bien sobre su marco teórico, simplemente se limita a decir que utilizará una herramienta
ya desarrollada en una de sus referencias (por ejemplo, la \emph{Compulsive Buying Scale}). Además de esto, me llama la atención de que tenga tan pocas citaciones según el índice Scopus (muchas, si no todas
en otros artículos de los mismos autores). Esto me produce cierta sospecha sobre la calidad del artículo, pues da la impresión de ser hecho de manera rápida y programática, en especial si se considera
la importancia que debería tener el método que en él se describe.


\bibliography{discretas_bd_1}
I. García, L. Jódar, P. Merello, F.-J. Santonja,
 A discrete mathematical model for addictive buying: Predicting the affected population evolution, 
\emph{Mathematical and Computer Modelling},
Volume 54, Issues 7–8,
 2011,
 Pages 1634-1637,
 ISSN 0895-7177,
 https://doi.org/10.1016/j.mcm.2010.12.012,
  (http://www.sciencedirect.com/science/article/pii/S0895717710005947)

\end{document}