% Created 2019-02-18 Mon 15:05
\documentclass[11pt]{article}
\usepackage[utf8]{inputenc}
\usepackage[T1]{fontenc}
\usepackage{fixltx2e}
\usepackage{graphicx}
\usepackage{longtable}
\usepackage{float}
\usepackage{wrapfig}
\usepackage{rotating}
\usepackage[normalem]{ulem}
\usepackage{amsmath}
\usepackage{textcomp}
\usepackage{marvosym}
\usepackage{wasysym}
\usepackage{amssymb}
\usepackage{hyperref}
\usepackage{amssymb}
\usepackage{amsfonts}
\usepackage{math}
\tolerance=1000
\author{Jonathan Ahumada}
\date{\today}
\title{Conjuntos}
\hypersetup{
  pdfkeywords={},
  pdfsubject={},
  pdfcreator={Emacs 25.3.1 (Org mode 8.2.10)}}
\begin{document}

\maketitle
\tableofcontents

\section{Conjuntos}
\label{sec-1}
Su definición no debe ser ambigua.
\subsection{Elementos}
\label{sec-1-1}
Letras minúsculas. También llamados miembros.
\[ a \in A\]
\[ a \notin A\]
\subsection{Conjuntos especiales}
\label{sec-1-2}
\[ \N = \{1,2,3,4,5,6, \dots{}\} \], los numeros naturales.
\Z = números enteros, positivos, cero o negaticos.
\Q = números racionales, de la forma $\backslash$(\frac$_{\text{m}}$\{n\}$\backslash$ | m,n$\backslash$ $\in$ \Z$\backslash$ y$\backslash$ n $\neq$ 0)
\R =  números reales, contiene a \Q y muchos números más, como $\pi$, \sqrt{3}, etc.

\subsection{Finitos}
\label{sec-1-3}
Son enumerables usando llaves \{\}. El etcétera matemático se puede 
usar cuando no hay ambiguedad. No obstante, muchas veces es imposible evitarla.
Por eso se deben describir los conjuntos por sus propiedades. 

\subsection{Descripción intensional}
\label{sec-1-4}
Se utiliza la notación:
\[\{\  :\ \} \]

Antes de los dos puntos, se indica la variable (por ejemplo x o n ) y después 
se dan sus propiedades. 
\[\{n : n \in \N y\ n\ es\ par\}\]

\subsection{Mediante la manera de obtener sus elementos}
\label{sec-1-5}

\[\{ n^2:\ n\ \in\ \N\}\]



\subsection{Son iguales}
\label{sec-1-6}
Si comparten los mismos elementos. 
Orden de enunmeracion es irrelevante.

\subsection{Subconjuntos}
\label{sec-1-7}
T es un subconjunto de S si \textbf{todo} elemento de T pertenece a S. 
Así que dos conjuntos son iguales si \( T \subseteq S y S \subseteq T\).

\subsubsection{Subconjunto de sí mismo}
\label{sec-1-7-1}

Como x $\in$ S implica que x $\in$ S, S \subseteq S. Un conjunto 
es subconjunto de sí mismo. Por eso la nooación \subseteq  en lugar de $\subset$.

Luego, decir $\subset$ es lo mismo que de decir que algo es un subconjunto de otro conjunto y que no es el mismo.
Si T $\subset$ S, decimos que T es \textbf{subconjunto propio} de S. 

El símbolo \subseteq es similar al $\le$

\subsection{Intervalos}
\label{sec-1-8}
Son subconjuntos especiales de \R.
\subsubsection{[ , ]}
\label{sec-1-8-1}
Los extremos estan incluidos --> \textbf{cerrados}.
\subsubsection{( , )}
\label{sec-1-8-2}
Los extremos estan excluidos --> \textbf{abiertos}.

\subsection{Conjunto vacío}
\label{sec-1-9}
$\emptyset$
% Emacs 25.3.1 (Org mode 8.2.10)
\end{document}